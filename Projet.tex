% Author: Nicolas Borie, Matthieu Josuat-Vergès, Samuele Giraudo
% Creation: may. 2018
% Modifications: may. 2018, june 2018

\documentclass[a4paper, 10pt]{article}

%%%%%%%%%%%%%%%%%%%%%%%%%%%%%%%%%%%%%%%%%%%%%%%%%%%%%%%%%%%%%%%%%%%%%%%%
%%%%%%%%%%%%%%%%%%%%%%%%%%%%%%%%%%%%%%%%%%%%%%%%%%%%%%%%%%%%%%%%%%%%%%%%
%%%%%%%%%%%%%%%%%%%%%%%%%%%%%%%%%%%%%%%%%%%%%%%%%%%%%%%%%%%%%%%%%%%%%%%%
\usepackage[utf8x]{inputenc}
\usepackage[french]{babel}
\usepackage{amsmath,amsfonts,amssymb,amsthm,shuffle}
\usepackage[T1]{fontenc}
%\usepackage[math]{anttor}
\usepackage{lmodern}

% Layout.
\usepackage[top=2.5cm,bottom=2.5cm,left=2.5cm,right=2.5cm]{geometry}

% Colors of hyperlinks.
\usepackage[dvipsnames]{xcolor}
\usepackage[hyperindex=true,frenchlinks=true,colorlinks=true,
citecolor=BrickRed,linkcolor=Gray,urlcolor=Gray,linktocpage,
pagebackref=true]{hyperref}

% Tikz.
\usepackage{tikz}
\usetikzlibrary{shapes}
\usetikzlibrary{fit}
\usetikzlibrary{decorations.pathmorphing}

% Misc.
\usepackage{mathtools}
\usepackage{dsfont}
\usepackage{wasysym}
\usepackage{stmaryrd}
\usepackage{cite}
\usepackage{subfig}
\usepackage{multirow}
\usepackage{enumitem}
\usepackage{multicol}
\usepackage{cancel}
\usepackage{eurosym}

%%%%%%%%%%%%%%%%%%%%%%%%%%%%%%%%%%%%%%%%%%%%%%%%%%%%%%%%%%%%%%%%%%%%%%%%
%%%%%%%%%%%%%%%%%%%%%%%%%%%%%%%%%%%%%%%%%%%%%%%%%%%%%%%%%%%%%%%%%%%%%%%%
%%%%%%%%%%%%%%%%%%%%%%%%%%%%%%%%%%%%%%%%%%%%%%%%%%%%%%%%%%%%%%%%%%%%%%%%
% Line space.
\linespread{1.0}
\renewcommand{\arraystretch}{1.4}

% Vertical space for equations.
\setlength{\abovedisplayskip}{5pt}
\setlength{\belowdisplayskip}{5pt}

% Alphabetic footnote marks.
\renewcommand{\thefootnote}{\alph{footnote}}

% To allow cutting equations in several pages.
\allowdisplaybreaks

% Numbering of equations.
\numberwithin{equation}{subsection}

% Depth of the table of contents.
\setcounter{tocdepth}{2}

% Indentation in the table of contents.
\makeatletter
\def\l@section{\@tocline{1}{3pt}{1pc}{5pc}{}}
\def\l@subsection{\@tocline{2}{2pt}{2pc}{5pc}{}}
\makeatother

% Environments definitions.
\newtheorem{Theorem}{Theorem}[subsection]
\newtheorem{Proposition}[Theorem]{Proposition}
\newtheorem{Lemma}[Theorem]{Lemma}

% Better comparison symbols.
\renewcommand{\leq}{\leqslant}
\renewcommand{\geq}{\geqslant}

%%%%%%%%%%%%%%%%%%%%%%%%%%%%%%%%%%%%%%%%%%%%%%%%%%%%%%%%%%%%%%%%%%%%%%%%
%%%%%%%%%%%%%%%%%%%%%%%%%%%%%%%%%%%%%%%%%%%%%%%%%%%%%%%%%%%%%%%%%%%%%%%%
%%%%%%%%%%%%%%%%%%%%%%%%%%%%%%%%%%%%%%%%%%%%%%%%%%%%%%%%%%%%%%%%%%%%%%%%
\title{Théorie d'Ehrhart et représentations polytopales de
l'associaèdre (TERPA)} % SG : joli, mais un peu long (2 lignes !!).
\title{{\bf P}olynômes d'{\bf E}hrhart des {\bf a}ssociaèdre{\bf s}
(PEAS)}
%% \keywords{Polytope, Associaèdre, Polynôme d'Ehrhart, théorie des représentations}
%% \subjclass[2010]{?}
\date{\today}
\author{Nicolas Borie \and Samuele Giraudo \and Matthieu Josuat-Verg\`es}
%% \address{\scriptsize Université Paris-Est, LIGM (UMR $8049$), CNRS,
%%    ENPC, ESIEE Paris, UPEM, F-$77454$, Marne-la-Vallée, France}
%% \email{nicolas.borie@u-pem.fr}
%% \email{matthieu.josuat-verges@u-pem.fr}
%% \email{samuele.giraudo@u-pem.fr}

%%%%%%%%%%%%%%%%%%%%%%%%%%%%%%%%%%%%%%%%%%%%%%%%%%%%%%%%%%%%%%%%%%%%%%%%
%%%%%%%%%%%%%%%%%%%%%%%%%%%%%%%%%%%%%%%%%%%%%%%%%%%%%%%%%%%%%%%%%%%%%%%%
%%%%%%%%%%%%%%%%%%%%%%%%%%%%%%%%%%%%%%%%%%%%%%%%%%%%%%%%%%%%%%%%%%%%%%%%

\newcommand{\N}{\mathbb{N}}
\newcommand{\Z}{\mathbb{Z}}
\newcommand{\Q}{\mathbb{Q}}
\newcommand{\K}{\mathbb{K}}

%%%%%%%%%%%%%%%%%%%%%%%%%%%%%%%%%%%%%%%%%%%%%%%%%%%%%%%%%%%%%%%%%%%%%%%%
%%%%%%%%%%%%%%%%%%%%%%%%%%%%%%%%%%%%%%%%%%%%%%%%%%%%%%%%%%%%%%%%%%%%%%%%
%%%%%%%%%%%%%%%%%%%%%%%%%%%%%%%%%%%%%%%%%%%%%%%%%%%%%%%%%%%%%%%%%%%%%%%%
\begin{document}

\maketitle
\vspace{-1.5cm}

%%%%%%%%%%%%%%%%%%%%%%%%%%%%%%%%%%%%%%%%%%%%%%%%%%%%%%%%%%%%%%%%%%%%%%%%
%%%%%%%%%%%%%%%%%%%%%%%%%%%%%%%%%%%%%%%%%%%%%%%%%%%%%%%%%%%%%%%%%%%%%%%%
%%%%%%%%%%%%%%%%%%%%%%%%%%%%%%%%%%%%%%%%%%%%%%%%%%%%%%%%%%%%%%%%%%%%%%%%
\section*{\center Informations pratiques}
Notre équipe se compose des membres suivants~:
\begin{center} \small
\begin{tabular}{|c|c|c|} \hline
    {\bf Membre} & {\bf Fonction} & {\bf Soutence de thèse}
        \\ \hline \hline
    N. Borie & Maître de conférences & 2011 \\ \hline
    S. Giraudo & Maître de conférences & 2011 \\ \hline
    M. Josuat-Vergès & Chargé de recherche & 2010 \\ \hline
\end{tabular}
\end{center}
\smallbreak

Évoluant tous les trois autour de la combinatoire algébrique, N. Borie
mène des recherches en liens avec la théorie des
représentations~\cite{Bor15}. S. Giraudo a la plus grande connaissance
du treillis de Tamari~\cite{Gir12} ayant produit plusieurs publications
à son sujet. Enfin M. Josuat-Vergès, porteur du projet, possède
l'expertise du plus grand spectre de domaines, ayant déjà contribué à la
théorie des polynômes d'Ehrhart~\cite{HJV16}, des polytopes ou encore à
la théorie des représentations.
\smallbreak

Le laboratoire gestionnaire du budget est le Laboratoire d'Informatique
Gaspard Monge (LIGM) de l'université Paris-Est à Marne-la-Vallée (UPEM).
Le contact administratif est Corinne Palescandolo, secrétaire au LIGM
(\href{mailto:corinne.palescandolo@u-pem.fr}
{\tt corinne.palescandolo@u-pem.fr}).
\smallbreak

\vspace{-1cm}

%%%%%%%%%%%%%%%%%%%%%%%%%%%%%%%%%%%%%%%%%%%%%%%%%%%%%%%%%%%%%%%%%%%%%%%%
%%%%%%%%%%%%%%%%%%%%%%%%%%%%%%%%%%%%%%%%%%%%%%%%%%%%%%%%%%%%%%%%%%%%%%%%
%%%%%%%%%%%%%%%%%%%%%%%%%%%%%%%%%%%%%%%%%%%%%%%%%%%%%%%%%%%%%%%%%%%%%%%%
\section*{\center Description du projet}

%%%%%%%%%%%%%%%%%%%%%%%%%%%%%%%%%%%%%%%%%%%%%%%%%%%%%%%%%%%%%%%%%%%%%%%%
%%%%%%%%%%%%%%%%%%%%%%%%%%%%%%%%%%%%%%%%%%%%%%%%%%%%%%%%%%%%%%%%%%%%%%%%
\paragraph{Contexte.}
En combinatoire algébrique, il est d'usage d'étudier des objets et des
constructions informatiques discrètes.  Ce qui différencie la
combinatoire algébrique des autres ramifications de la combinatoire est
qu'elle utilise des outils provenant de l'algèbre. L'un des objectifs
consiste à  comprendre le plus finement possible le comportement des
constructions combinatoires courantes (mots, arbres, graphes, chemins,
figures, {\em etc.}) dans le but de les utiliser avec le maximum
d'efficacité lors des applications algorithmiques.
\smallbreak

Parmi ces objets, le treillis de Tamari apporte son lot de questions
ouvertes. Il s'agit d'une structure d'ordre partiel combinatoire qui
fait intervenir les arbres binaires et leur opération dite de rotation.
Cette structure a été introduite historiquement par Dov
Tamari~\cite{Tam62} dans un contexte algébrique et a été depuis
redécouverte dans un contexte algorithmique~\cite{Knu98}, notamment à
travers les liens qu'il entretient avec les arbres binaires
équilibrés~\cite{AVL62}. Ce treillis est l'un des objets centraux de la
thèse de Samuele Giraudo, membre de ce projet~\cite{Gir12}. Des travaux
récents ont contribué à sa compréhension. Parmi eux, l'un de nature
combinatoire~\cite{CP15} a été amené par Viviane Pons%
\footnote{Laboratoire de recherche en informatique de
l'université d'Orsay (LRI).},
et un autre, de nature géométrique~\cite{PS15}, a été poursuivi par
Vincent Pilaud%
\footnote{Laboratoire d'informatique de l'école polytechnique (LIX).}.
\smallbreak

\vspace{-.25cm}

%%%%%%%%%%%%%%%%%%%%%%%%%%%%%%%%%%%%%%%%%%%%%%%%%%%%%%%%%%%%%%%%%%%%%%%%
%%%%%%%%%%%%%%%%%%%%%%%%%%%%%%%%%%%%%%%%%%%%%%%%%%%%%%%%%%%%%%%%%%%%%%%%
\paragraph{Sujet de recherche.}
Un fait remarquable du treillis de Tamari est qu'il est possible de le
réaliser comme une figure géométrique en haute dimension, connue sous le
nom d'associaèdre~\cite{Lod04}. L'objectif de ce projet est d'obtenir
une description combinatoire des polynômes d'Ehrhart associé aux
différentes réalisations géométriques du treillis de Tamari. À tout
polytope correspond un polynôme d'Ehrhart qui encode un certain nombre
d'information sur celui-ci, comme son volume. Notre projet est de
calculer les polynômes d'Ehrhart des différentes réalisations
géométriques du treillis de Tamari et de s'en servir comme tremplin à
des découvertes nouvelles de ses propriétés algorithmiques et
combinatoires. Le porteur du projet Matthieu Josuat-Vergès est déjà
familier avec la théorie d'Ehrhart~\cite{HJV16} et en a d'ailleurs déjà
extrait des liens avec la théorie des représentations.
\smallbreak

\vspace{-.25cm}

%%%%%%%%%%%%%%%%%%%%%%%%%%%%%%%%%%%%%%%%%%%%%%%%%%%%%%%%%%%%%%%%%%%%%%%%
%%%%%%%%%%%%%%%%%%%%%%%%%%%%%%%%%%%%%%%%%%%%%%%%%%%%%%%%%%%%%%%%%%%%%%%%
\paragraph{Méthode de travail.}
Pour mener à bien cet objectif, un premier travail d'exploration
informatique sera nécessaire. Calculer les polynômes d'Ehrhart est déjà
un défi informatique, aussi la perspective de ce projet permet
d'envisager une délégation des calculs d'exploration sur machine
distante (cloud computing). Il s'en suivra alors une étude fine de ces
futurs résultats pour en proposer la meilleure description. Nous
espérons aussi mettre en lumière des interprétations de ces résultats en
terme de théorie des représentations comme le porteur du projet à pu
constater dans~\cite{HJV16} dans un autre contexte.
\smallbreak

\vspace{-.25cm}

%%%%%%%%%%%%%%%%%%%%%%%%%%%%%%%%%%%%%%%%%%%%%%%%%%%%%%%%%%%%%%%%%%%%%%%%
%%%%%%%%%%%%%%%%%%%%%%%%%%%%%%%%%%%%%%%%%%%%%%%%%%%%%%%%%%%%%%%%%%%%%%%%
\paragraph{Résultats attendus.}
Il est raisonnable de viser à une soumission pour publication de ces
résultats dans un délai d'un an environ. Nous prévoyons une soumission
d'un résumé étendu dans la conférence internationale FPSAC%
\footnote{Formal PowerSeries and Algebraic Combinatorics
(\url{http://fpsac.org/confs/fpsac-2019/}).}
ainsi qu'une soumission d'une version complète de nos résultats dans un
journal de combinatoire algébrique (à déterminer encore précisément
selon la tournure de nos travaux).
\smallbreak

\vspace{-1cm}

%%%%%%%%%%%%%%%%%%%%%%%%%%%%%%%%%%%%%%%%%%%%%%%%%%%%%%%%%%%%%%%%%%%%%%%%
%%%%%%%%%%%%%%%%%%%%%%%%%%%%%%%%%%%%%%%%%%%%%%%%%%%%%%%%%%%%%%%%%%%%%%%%
%%%%%%%%%%%%%%%%%%%%%%%%%%%%%%%%%%%%%%%%%%%%%%%%%%%%%%%%%%%%%%%%%%%%%%%%
\section*{\center Projet d'utilisation du financement}
Nous envisageons la répartition suivante du budget~:
\begin{itemize}
    \item {\bf achat de livres}, pour la constitution d'une petite
    bibliothèque sur le sujet (de 300 € à 500 €)~;

    \item {\bf souscription à des services de cloud computing} comme
    {\sc CoCalc} (\url{https://cocalc.com/}) ou {\sc Google Cloud}
    (\url{https://cloud.google.com/}) pour réaliser des calculs
    et procéder à des expérimentations informatiques (de 1300 € à 1600 €)%
    \footnote{Cet élément du budget est sujet à discrétion et nous
    sommes disposés à le supprimer si ce type de dépense n'est pas
    couvert par le soutien.}%
    ~;

    \item {\bf organisation de deux rencontres}, chacune sur une
    journée. La
    première se fera lors du lancement du projet, en fin d'année 2018.
    L'objectif sera d'inviter des chercheurs principalement de
    laboratoires d'Île de France (Vincent Pilaud, Viviane Pons,
    Bérénice Delcroix-Oger%
    \footnote{Institut de Recherche en Informatique Fondamentale
    (IRIF).})
    afin de former un groupe de travail
    (exposés, sessions de discussions et de recherche). Le budget
    nous servira à couvrir les déplacements et les repas. La
    seconde journée est prévue quelques mois avant la fin du
    financement, vers septembre 2019, constituée des mêmes éléments et
    ayant pour but de faire un point sur nos avancées. Nous
    prévoyons une dépense de 1500 € à 2000 € pour les deux journées~;

    \item {\bf invitation de collaborateurs}, qui peuvent être l'un de
    ceux cités préalablement, ou encore d'autres attirés par nos
    journées. Il est difficile de prévoir exactement un budget pour cela
    mais nous estimons que ces invitations, qui incluent voyage et
    logement, chiffreront pour entre 2900 € et 3300 €~;

    \item {\bf participation à des conférences}, en vue de présenter
    nos résultats à des conférences internationales. La prochaine
    conférence pertinente à notre projet est FPSAC
     et se déroula en Slovénie.
    Nous pouvons ajouter à cela la prise en charge de déplacements
    éventuels des membres du projets pour mener à bien des collaborations
    éventuelles.
    Nous prévoyons pour cela entre 4500 € et 5500 € de frais.
\end{itemize}
\smallbreak

Cela correspond à un total d'environ 11700 €.
\smallbreak

%%%%%%%%%%%%%%%%%%%%%%%%%%%%%%%%%%%%%%%%%%%%%%%%%%%%%%%%%%%%%%%%%%%%%%%%
%%%%%%%%%%%%%%%%%%%%%%%%%%%%%%%%%%%%%%%%%%%%%%%%%%%%%%%%%%%%%%%%%%%%%%%%
%%%%%%%%%%%%%%%%%%%%%%%%%%%%%%%%%%%%%%%%%%%%%%%%%%%%%%%%%%%%%%%%%%%%%%%%
\bibliographystyle{plain}
\begin{footnotesize}
\bibliography{Projet}
\end{footnotesize}

\end{document}
