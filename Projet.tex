% Author: Nicolas Borie, Matthieu Josuat-Vergès, Samuele Giraudo
% Creation: may. 2018
% Modifications: may. 2018, june 2018

\documentclass[a4paper, 10pt]{article}

%%%%%%%%%%%%%%%%%%%%%%%%%%%%%%%%%%%%%%%%%%%%%%%%%%%%%%%%%%%%%%%%%%%%%%%%
%%%%%%%%%%%%%%%%%%%%%%%%%%%%%%%%%%%%%%%%%%%%%%%%%%%%%%%%%%%%%%%%%%%%%%%%
%%%%%%%%%%%%%%%%%%%%%%%%%%%%%%%%%%%%%%%%%%%%%%%%%%%%%%%%%%%%%%%%%%%%%%%%
\usepackage[utf8x]{inputenc}
\usepackage[french]{babel}
\usepackage{amsmath,amsfonts,amssymb,amsthm,shuffle}
\usepackage[T1]{fontenc}
\usepackage[math]{anttor}

% Layout.
\usepackage[top=2.5cm,bottom=2.5cm,left=2.5cm,right=2.5cm]{geometry}

% Colors of hyperlinks.
\usepackage[dvipsnames]{xcolor}
\usepackage[hyperindex=true,frenchlinks=true,colorlinks=true,
citecolor=BrickRed,linkcolor=Gray,urlcolor=Gray,linktocpage,
pagebackref=true]{hyperref}

% Tikz.
\usepackage{tikz}
\usetikzlibrary{shapes}
\usetikzlibrary{fit}
\usetikzlibrary{decorations.pathmorphing}

% Misc.
\usepackage{mathtools}
\usepackage{dsfont}
\usepackage{wasysym}
\usepackage{stmaryrd}
\usepackage{cite}
\usepackage{subfig}
\usepackage{multirow}
\usepackage{enumitem}
\usepackage{multicol}
\usepackage{cancel}
\usepackage{eurosym}

%%%%%%%%%%%%%%%%%%%%%%%%%%%%%%%%%%%%%%%%%%%%%%%%%%%%%%%%%%%%%%%%%%%%%%%%
%%%%%%%%%%%%%%%%%%%%%%%%%%%%%%%%%%%%%%%%%%%%%%%%%%%%%%%%%%%%%%%%%%%%%%%%
%%%%%%%%%%%%%%%%%%%%%%%%%%%%%%%%%%%%%%%%%%%%%%%%%%%%%%%%%%%%%%%%%%%%%%%%
% Line space.
\linespread{1.15}
\renewcommand{\arraystretch}{1.4}

% Vertical space for equations.
\setlength{\abovedisplayskip}{5pt}
\setlength{\belowdisplayskip}{5pt}

% Alphabetic footnote marks.
\renewcommand{\thefootnote}{\alph{footnote}}

% To allow cutting equations in several pages.
\allowdisplaybreaks

% Numbering of equations.
\numberwithin{equation}{subsection}

% Depth of the table of contents.
\setcounter{tocdepth}{2}

% Indentation in the table of contents.
\makeatletter
\def\l@section{\@tocline{1}{3pt}{1pc}{5pc}{}}
\def\l@subsection{\@tocline{2}{2pt}{2pc}{5pc}{}}
\makeatother

% Environments definitions.
\newtheorem{Theorem}{Theorem}[subsection]
\newtheorem{Proposition}[Theorem]{Proposition}
\newtheorem{Lemma}[Theorem]{Lemma}

% Better comparison symbols.
\renewcommand{\leq}{\leqslant}
\renewcommand{\geq}{\geqslant}

%%%%%%%%%%%%%%%%%%%%%%%%%%%%%%%%%%%%%%%%%%%%%%%%%%%%%%%%%%%%%%%%%%%%%%%%
%%%%%%%%%%%%%%%%%%%%%%%%%%%%%%%%%%%%%%%%%%%%%%%%%%%%%%%%%%%%%%%%%%%%%%%%
%%%%%%%%%%%%%%%%%%%%%%%%%%%%%%%%%%%%%%%%%%%%%%%%%%%%%%%%%%%%%%%%%%%%%%%%
\title{Théorie d'Ehrhart et représentations polytopales de
l'associaèdre (TERPA)}
%% \keywords{Polytope, Associaèdre, Polynôme d'Ehrhart, théorie des représentations}
%% \subjclass[2010]{?}
\date{\today}
\author{Nicolas Borie \and Matthieu Josuat-Verg\`es \and Samuele Giraudo}
%% \address{\scriptsize Université Paris-Est, LIGM (UMR $8049$), CNRS,
%%    ENPC, ESIEE Paris, UPEM, F-$77454$, Marne-la-Vallée, France}
%% \email{nicolas.borie@u-pem.fr}
%% \email{matthieu.josuat-verges@u-pem.fr}
%% \email{samuele.giraudo@u-pem.fr}

%%%%%%%%%%%%%%%%%%%%%%%%%%%%%%%%%%%%%%%%%%%%%%%%%%%%%%%%%%%%%%%%%%%%%%%%
%%%%%%%%%%%%%%%%%%%%%%%%%%%%%%%%%%%%%%%%%%%%%%%%%%%%%%%%%%%%%%%%%%%%%%%%
%%%%%%%%%%%%%%%%%%%%%%%%%%%%%%%%%%%%%%%%%%%%%%%%%%%%%%%%%%%%%%%%%%%%%%%%

\newcommand{\N}{\mathbb{N}}
\newcommand{\Z}{\mathbb{Z}}
\newcommand{\Q}{\mathbb{Q}}
\newcommand{\K}{\mathbb{K}}

%%%%%%%%%%%%%%%%%%%%%%%%%%%%%%%%%%%%%%%%%%%%%%%%%%%%%%%%%%%%%%%%%%%%%%%%
%%%%%%%%%%%%%%%%%%%%%%%%%%%%%%%%%%%%%%%%%%%%%%%%%%%%%%%%%%%%%%%%%%%%%%%%
%%%%%%%%%%%%%%%%%%%%%%%%%%%%%%%%%%%%%%%%%%%%%%%%%%%%%%%%%%%%%%%%%%%%%%%%
\begin{document}

\maketitle

%%%%%%%%%%%%%%%%%%%%%%%%%%%%%%%%%%%%%%%%%%%%%%%%%%%%%%%%%%%%%%%%%%%%%%%%
%%%%%%%%%%%%%%%%%%%%%%%%%%%%%%%%%%%%%%%%%%%%%%%%%%%%%%%%%%%%%%%%%%%%%%%%
%%%%%%%%%%%%%%%%%%%%%%%%%%%%%%%%%%%%%%%%%%%%%%%%%%%%%%%%%%%%%%%%%%%%%%%%
\section{Description du projet}

Nicolas : TITRE PLUS VULGARISÉ ???? Faites vous plaisir !

Nicolas : Doit-on utiliser la première ou la troisième personne du pluriel dans ce document ?
(Nous comptons faire cela... OU BIEN Les porteurs du projet comptent faire cela...)


La combinatoire algébrique étudie des objets et des constructions
informatiques discrètes. L'objectif consiste à comprendre le plus
finement le comportement des constructions combinatoires (arbres,
graphes, chemins, figures) dans le but de les utiliser le plus
efficacement possible lors des applications algorithmiques. C'est une
recherche théorique qui visent à étudier la faisabilité, optimiser les
utilisations et établir un état de l'art de la manière dont doivent
être utilisés ces objets centraux en informatique. Les arbres
notamment pour l'organisation numérique des données, les graphes par
exemple pour les transports, etc.



Parmi ces objets, le treillis de Tamari pose encore de nombreuses
questions ouvertes. De récentes avancées ont été proposées par Viviane
Pons~\cite{MR3345297} du laboratoire de recherche (LRI) en
informatique de l'université d'Orsay. L'objet a aussi été étudié
géométriquement par Vincent Pilaud~\cite{MR3327085} du laboratoire
d'informatique de l'école polytechnique (LIX). Ce treillis est aussi
un des objets pilier de la thèse de Samuele Giraudo, membre de ce
projet~\cite{MR2887627}. C'est un objet combinatoire pouvant être
appliqué à la gestions de priorités, à la recherche de données, à
l'ordonnancement d'opérations (Nicolas : Soyez vendeur et meilleur que
moi, je ne suis pas expert Tamari labelled).



L'objectif de ce projet est d'obtenir une description combinatoire des
polynômes d'Ehrhart associé aux réalisation polytopales de
l'associaèdre. Ces réalisations, différentes et nombreuses, réalise
géométriquement le treillis de Tamari dont nous devrions renforcer la
compréhension. Le porteur du projet Matthieu Josuat-Vergès est déjà
familier avec la théorie d'Ehrhart~\cite{MR3484760} et en a d'ailleurs
déjà extrait des liens avec la théorie des représentations.



Pour mener à bien cet objectif, un premier travail d'exploration
informatique sera nécessaire. Calculer les polynômes d'Ehrhart est
déjà un challenge informatique, aussi la perspective de ce projet
permet d'envisager une délégation des calculs d'exploration sur
machine distance (Cloud computing). Il s'en suivra alors une étude
fine de ces futurs résultats pour en proposer la meilleure
description. Les membres espèrent aussi mettre en lumière des
interprétations de ces résultats en terme de théorie des
représentation comme le porteur du projet à pu constater
dans~\cite{MR3484760} dans un autre contexte.



Il est raisonnable de penser qu'une soumission pour publication de ces
résultats à une échéance de un an est possible. Cela fait déjà parti
des objectifs que l'équipe s'est fixés si le projet est accepté.



L'équipe sera composé de trois membres. Nicolas Borie (thèse en 2011),
maître de conférence au laboratoire d'informatique Gaspard Monge
(LIGM) de l'université Paris Est à Marne-La-Vallée (UPEM). Samuele
Giraudo (thèse en 2011), maître de conférence habilité aussi au LIGM
et son porteur Matthieu Josuat-Vergès (thèse en 1515) chargé de
recherche au CNRS, actuellement en détachement à l'Institut de
Recherche en Informatique Fondamentale (IRIF) de l'université Paris~7.
Évoluant tous les trois autour de la géométrie algébrique, Borie
propose des recherches en liens avec la théorie des
représentations~\cite{MR3448031}, Giraudo a la plus grande
connaissance du treillis de Tamari~\cite{MR2887627} ayant produit
plusieurs publications à son sujet enfin Josuat-Vergès possède le plus
grand spectre de domaines, ayant déjà publié à propos de théorie
d'Ehrhart, de polytopes ou encore de théorie des représentations.



Nous souhaitons solliciter le DIM émergent RFSI pour un budget à hauteur de xxxx.


%Description de l'utilisation du budget (Nicolas : Désolé, je n'ai pas noté la liste lors de notre dernière réunion...)
%livres \\
%cloud computing (capacité de calculs distantes) \\
%Mentionner des meetings lors desquels nous ferons parler les chercheurs franciliens à propos de Tamari/associaèdre. \\
%Présentation des résultats lors d'une conférence \\



Le laboratoire d'informatique Gaspard Monge (LIGM) de l'université
Paris-Est à Marne-la-Vallée (UPEM) sera gestionnaire du
budget. Corinne Palescandolo (\texttt{corinne.palescandolo@u-pem.fr})
sera le contact administratif pour la gestion du projet.

%%%%%%%%%%%%%%%%%%%%%%%%%%%%%%%%%%%%%%%%%%%%%%%%%%%%%%%%%%%%%%%%%%%%%%%%
%%%%%%%%%%%%%%%%%%%%%%%%%%%%%%%%%%%%%%%%%%%%%%%%%%%%%%%%%%%%%%%%%%%%%%%%
%%%%%%%%%%%%%%%%%%%%%%%%%%%%%%%%%%%%%%%%%%%%%%%%%%%%%%%%%%%%%%%%%%%%%%%%
\section{Projet d'utilisation du financement}
Nous envisageons la répartition suivante du budget~:
\begin{itemize}
    \item {\bf achat de livres}, pour la constitution d'une petite
    bibliothèque sur le sujet (de 200 € à 400 €)~;
    \smallbreak

    \item {\bf souscription à des services de cloud computing} comme
    {\sc CoCalc} (\url{https://cocalc.com/}) pour réaliser des calculs
    et procéder à des expérimentations informatiques (1300 €)%
    \footnote{Cet élément du budget est sujet à discrétion et nous
    sommes disposés à le supprimer si ce type de dépense n'est pas
    couvert par le soutien.}%
    ~;
    \smallbreak

    \item {\bf organisation de deux rencontres}, chacune sur une
    journée. La
    première se fera lors du lancement du projet, en fin d'année 2018.
    L'objectif sera d'inviter des chercheurs principalement de
    laboratoires d'Île de France (Vincent Pilaud, Viviane Pons,
    Bérénice Delcroix-Oger) afin de former un groupe de travail
    (exposés, sessions de discussions et de recherche). Le budget
    nous servira à couvrir les déplacements et les repas. La
    seconde journée est prévue quelques mois avant la fin du
    financement, vers septembre 2019, constituée des mêmes éléments et
    ayant pour but de faire un point sur nos avancées. Nous
    prévoyons une dépense de 500 € à 800 € pour les deux journées~;
    \smallbreak

    \item {\bf invitation de collaborateurs}, qui peuvent être l'un de
    ceux cités préalablement, ou encore d'autres attirés par nos
    journées. Il est difficile de prévoir exactement un budget pour cela
    mais nous estimons que ces invitations, qui incluent voyage et
    logement, chiffreront pour entre 2000 € et 3000 €~;
    \smallbreak

    \item {\bf participation à des conférences}, en vue de présenter
    nos résultats à des conférences internationales. La prochaine
    conférence pertinente à notre projet est FPSAC
    (\url{http://fpsac.org/confs/fpsac-2019/}) et se déroula en Slovénie.
    Nous pouvons ajouter à cela la prise en charge de déplacements
    éventuels des membres du projets pour mener à bien des collaborations
    éventuelles.
    Nous prévoyons pour cela entre 3500 € et 4500 € de frais~;
\end{itemize}
\medbreak

Cela correspond à un total d'environ 9000 €.

{\em SG~: ce n'est pas assez. On peut être plus gourmands sur le
cloud computing (voir Amazon et Google par exemple).}


%%%%%%%%%%%%%%%%%%%%%%%%%%%%%%%%%%%%%%%%%%%%%%%%%%%%%%%%%%%%%%%%%%%%%%%%
%%%%%%%%%%%%%%%%%%%%%%%%%%%%%%%%%%%%%%%%%%%%%%%%%%%%%%%%%%%%%%%%%%%%%%%%
%%%%%%%%%%%%%%%%%%%%%%%%%%%%%%%%%%%%%%%%%%%%%%%%%%%%%%%%%%%%%%%%%%%%%%%%
\bibliographystyle{alpha}
\bibliography{Projet}

\end{document}
